\chapitre{Mise en place du projet}
La mise en place de ce projet peut parraitre relativement simple pour un habitué mais s'avère trés fastidieuse pour une personne novice.
La partie suivant a pour objectif de détailler les différentes étapes de la mise en place de ce projet.
%Afin de rendre accessible ce projet au plus grand nombre. 
Pour mettre en place un reseau domotique, comme celui exposé dans ce rapport, vous aurait besoins du matériel suivant :

begin{itemize}
	\item D'un  ordinateur. Cette ordinateur vous servira à configurer la raspberry et  les différents capteurs grace à un site internet.
	\item D'une raspberry-pi I ou II. 
	\item D'un lecteur de carte SD (ou si vous avait opté pour une raspeberry-pi II un lecteur de micro carte SD).
	\item D'une antenne Z-Wave
	\item De différents capteurs disposant de la technologie Zwave. Nous avons choisi, pour tester notre projet, des capteurs venant de l'entreprise FIBARO. Les capteur choisit sont floodSensor, WallPlug et Smoke Sensor.
\end{itemize}

\section{Raspberry}
Une raspberry pi est un petit ordinateur avec un prix trés abordable (autour de 30 euros).
Avant toutes chose vous avez besoin de configurer sa memoire. Concrètement vous allez creer sur la carte SD (ou micro SD) un systeme de fichier.
Ce système de fichier permettra à la raspberry PI de démarrer, de sauvegarder les paramètres dont elle a besoin dans différents fichiers et de permettre à l'utilisateur de stoquer des données.
Afin d'effectuer cette action, vous devez télécharger l'ISO officiel 
Pour des utilisateurs beaucoup plus expirimentés vous pouvez vous même creer votre système de fichier grace à l'ouitl buildroot.mettre la carte SD ( ou la micro carte SD) dans le lecteur 

