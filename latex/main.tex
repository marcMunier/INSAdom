%TODO synthèse
%TODO glossaire et bibliographie etc ..

%%%%%%%%%%%%%%%%%%%%%%%%%%%%%%%%%%%%%%%%%%%%%%%%%%%%%%%%%
%                      Préambule                        %
%%%%%%%%%%%%%%%%%%%%%%%%%%%%%%%%%%%%%%%%%%%%%%%%%%%%%%%%%

%%%%%%%%%%%%%%%%%%%%%%%%%%%%%%%%%%%%%%%%%%%%%%%%%%%%%%%%%%%%%%%%%
%                      Préambule type                          %
%%%%%%%%%%%%%%%%%%%%%%%%%%%%%%%%%%%%%%%%%%%%%%%%%%%%%%%%%%%%%%%%



\def\changemargin#1#2{\list{}{\rightmargin#2\leftmargin#1}\item[]}
\let\endchangemargin=\endlist 

%%%%%%%%%%%%%%%%%% Classe dudit Document %%%%%%%%%%%%%%%%%%%%%%%
\documentclass[a4paper,12pt,openany]{report} % sert à définir des propriétés e base sur les articles
% Autre paramètres connnus: 	*book => pour de vrais livres
%				*article pour des artiles dans des revues scientifiques, des présentations, des rapports courts, des documentations, des invitation etc....
%				*report por des rapports plus long contenant plusieurs chapitres, des petits livres, des thèses
%				*Slides pour des transparants
%				*foiltex
% Option: Xpt taille de la police,  4apaper/letterpaper/a5paper/b5paper/executivepaper, fleqn, leqno, titlepage/notitlepage indique si une nouvelle page doit être commencée après le titre du document, twocolumn,twoside/oneside,



%%%%%%%%%%%%%%%%%%%%%%%% Package %%%%%%%%%%%%%%%%%%%%%%%%%%%%%%%
\usepackage{doc}% permet de documenter des programmes (cf doc.dtx]
\usepackage{exscale} % fournit des version de taille de caractère que LaTeX va utiliser (cf ltexscale.dtx)
\usepackage{fontenc} % spécifie le codage des police de caractère que LaTeX va utiliser (cf ltoutenc.dtx)
%\usepackage[latin]{inputenc} %Autorise les caractères acentués
\usepackage{ifthen} % fournit des commandes de conditions (cf ifthen.dtx)
\usepackage{latexsym} %permet l'utilisation de la police des symboles LaTeX
\usepackage{makeidx} %fournit des commandes pour réaliser un index
\usepackage{syntonly} % analyse le doc sans le formater
\usepackage[T1]{fontenc}
\usepackage[utf8x]{inputenc} 
%\usepackage[utf8]{inputenc}% permet de spécifier le codeage des caractère utiliser dans le source.
\usepackage[english,francais]{babel}
\usepackage{xcolor}
%\usepackage{asmath} %Fonctions servant à l'écriture mathématique
\usepackage{xspace} %Fonctions permettant d'introduire des espaces
\usepackage{xargs} %Permet d'utiliser de définir plus simplement des fonctions
\usepackage{trace}%Permet d'utiliser des traces de debbug
\usepackage{show2e}% debbug
\usepackage[pdftex]{graphicx}
%\usepackage{hyperref}
\usepackage{url}
\usepackage{chngcntr}
\counterwithout{figure}{chapter}

%%%%%%%%%%%%%%%%% Style du pied de page %%%%%%%%%%%%%%%%%%%%%%%%

\pagestyle{plain}% imprime le numéro de page au milieu du pied de page
%\pagestyle{heading}%imprime le titre du chapitre courant et le numéro dans l'en tête et laisse le pied de page vide.
%\pagestyle{empty}% laisse l'en tête et le pied de page vide


%Nota bene: On peut changer le pied de page en court en utilisant \thispagestyle


%%%%%%%%%%%%%%%%% Césure   %%%%%%%%%%%%%%%%%%%%%%%%

\hyphenation{FORTRAN}
\hyphenation{An-ti-cons-ti-tu-tion-nel-le-ment}

%%%%%%%%%%%%%%%% Citation Environement %%%%%%%%%%%%


% un boite pour l ’auteur de la citation
\newsavebox{\auteurcitation}
\newsavebox{\boitecitation}
\definecolor{coulcitation}{rgb}{0.60,0.70,0.90}%

\newenvironment{terminal}[1]{% clause begin
% on sauve l ’argument 1 pour l ’auteur
	\savebox{\auteurcitation}{#1}%
	\begin{lrbox}{\boitecitation}


	\begin{minipage}{.8\linewidth}
	\small\slshape%
}% on passe en petit et penché
{% clause end : on pousse l ’auteur de la citation à droite
	\end{minipage}
	\end{lrbox}
	\begin{center}
	\colorbox{coulcitation}{\usebox{\boitecitation}}
	\end{center}
}



%
\input ./preambule.tex %sans saut de ligne


%%%%%%%%%%%%%%Titre%%%%%%%%%%%%%%%%%%%%%%%%%%%%%%%%%%%%
\title{ Projet de 3 année à l'INSA Centre Val de Loire } % Titre du document
\author{BAIZ Mamoune && BAIZ Mamoune} % Auteur
\date{\today} % Date de création


%%%%%%%%%%%%%%%%%%%%%%%Debut Doc%%%%%%%%%%%%%%%%%%%%%%%
\begin{document} % Début du document
%%%%%%%%%%%%%%%%%%%%%%Page de Garde%%%%%%%%%%%%%%%%%%%%
{
\begin{titlepage}
  \begin{sffamily}
  \begin{center}

    % Upper part of the page. The '~' is needed because \\
    % only works if a paragraph has started.
    \includegraphics[scale=0.5]{Images/png/insa_logo.png}~\\[1.5cm]

    \textsc{\LARGE Institut National des Sciences Appliquées  Centre Val de Loire }\\[2cm]

    \textsc{\Large Rapport de projet 3\up{ème} année option 2SU}\\[1.5cm]

    % Title
%    \HRule \\[0.4cm]
    { \huge \bfseries Domotique\\[0.4cm] }

%    \HRule \\[2cm]
    \includegraphics[scale=0.5]{Images/jpg/domotic.jpg}
    \\[2cm]

    % Author and supervisor
    \begin{minipage}{0.4\textwidth}
      \begin{flushleft} \large
        BAIZ Mamoune et MUNIER Marc \\
        Promo 2016\\
      \end{flushleft}
    \end{minipage}
    \begin{minipage}{0.4\textwidth}
      \begin{flushright} \large
        \emph{Encadrant :} M. Briffaut\\
      \end{flushright}
    \end{minipage}

    \vfill

    % Bottom of the page
    {\large 9 Février 2016}

  \end{center}
  \end{sffamily}
\end{titlepage}
}
%%%%%%%%%%%%%%%%%%%%%%%Page Vierge%%%%%%%%%%%%%%%%%%%%
\newpage



%%%%%%%%%%%%%%%%%%%%% Remerciment %%%%%%%%%%%%%%%%%%%%
\newpage
\section*{Remerciement}
Au terme de notre formation à L’INSA Centre Val de Loire, et tout au long de ce projet il est nécessaire de remercier tous mes professeurs, ainsi que tout le corps professoral et administratif de notre établissement, auxquels je tiens à rendre hommage pour leurs efforts prodigieux qu’ils n’ont cessé de fournir afin que nous puissions, mes collèges et moi, avoir une formation solide et rigoureuse ; pour leur encadrement tout au long de cette année, et pour leur disponibilité permanente. Je n’oublierais pas de remercier spécialement M.Briffaut pour son soutien tout au long du projet, ainsi que ces précieux cours de "domotique" sur lesquelles nous avons abouti à réaliser ce projet.\newline

%marc: ok pour moi peut etre revoir la forme
\clearpage
%%%%%%%%%%%%%%%%%%%%%Résumé%%%%%%%%%%%%%%%%%%%%%%%%%%%
\section*{Résumé}

   Votre résumé commence ici...
   ...


%%%%%%%%%%%%%%%%%%%%%table des matières%%%%%%%%%%%%%%%
\newpage
\tableofcontents
\clearpage

%%%%%%%%%%%PLAN
%
%\section*{Introduction} %( comment on a fait ce projet, en quoi il consiste, les enjeux )
%Mamoune
\section*{Presentation du résultat final}
\subsection*{les outils} ( raspberry + capteur) 
Afin de réaliser ce projet, quelques outils et éléments nous ont été indispensables. Tout d'abord, la centrale qui contrôle tout le système de domotique: La Raspberry Pi.

Le Raspberry Pi est un nano-ordinateur monocarte à processeur ARM conçu par le créateur de jeux vidéo David Braben, dans le cadre de sa fondation Raspberry Pi.Cet ordinateur, qui a la taille d'une carte de crédit, est destiné à encourager l'apprentissage de la programmation informatique2 ; il permet l'exécution de plusieurs variantes du système d'exploitation libre GNU/Linux et des logiciels compatibles à plusieurs protocole de domotique. Il est fourni nu (carte mère seule, sans boîtier, alimentation, clavier, souris ni écran) dans l'objectif de diminuer les coûts et de permettre l'utilisation de matériel de récupération. Cet outil est alors la pièce maîtresse de notre projet domotique. 
\subsubsection{Description des composants de la raspberry pi 2}
-Environnement 	Linux (Debian, Fedora et ArchLinux), RISC OS , Windows IOT


-Système d'exploitation 	Linux (Raspbian, Pidora, et Arch Linux ARM gentoo), RISC OS, FreeBSD, NetBSD, Windows 10 IoT (uniquement compatible avec le Raspberry Pi 2), Plan 9


-Alimentation 	Micro-USB 5 V


-Processeur 	Broadcom BCM2835 - ARM1176JZF-S 700 MHz (modèle 1) ou 1 GHz (Modèle Zero)1


-Broadcom BCM2836 - Cortex-A7 900 MHz (modèle 2)


-Stockage 	Carte SD (A, B), Carte microSD (A+,B+,2)


-Mémoire 	256 Mo (modèle A et A+)
256 Mo (modèle B rev 1)
512 Mo (modèle B rev 2 et B+)
1 Go (modèle 2)


-Carte graphique 	Broadcom VideoCore IV1,


-Connectivité 	USB, Ethernet (modèle B, B+ ,2) (RJ45), HDMI, RCA, Jack 3,5 mm, Micro USB


-Dimensions 	85,60 mm × 53,98 mm × 17 mm (A, B, B+),
65 mm × 53,98 mm × 17 mm (A+),
65 mm × 30 mm × 5 mm (Zero)


-Masse 	44,885 g (A, B, B+), 23 g (A+)

\subsubsection{Les capteurs utilisés}
Afin de savoir les données concernant le milieu ambiant contournant la centrale de domotique (la raspberry pi2), il est nécessaire d'utiliser des capteurs pour capturer les données que nous souhaitons, mais aussi d'une antenne posée sur la raspberry analysant les données qu'elle reçoit depuis les capteurs.

\subsubsubsection{Antenne utilisée}
Nous avons utilisée une antenne Z-Wave. Z-Wave est un protocol radio conçu pour la domotique, facilement intégré avec la raspberry pi2. Z-Wave fonctionne dans la gamme de fréquences sous-gigahertz, qui dépend des régions (868 MHz en Europe, 908 MHz aux US, et d'autres fréquences suivant les bandes ISM des régions). La portée est d'environ 50 m (davantage en extérieur, moins en intérieur). La technologie utilise la technologie du maillage (mesh) pour augmenter la portée et la fiabilité.

-Vulnérabilité:
Z-Wave se base sur une seule plage de fréquence et est donc vulnérable à un brouilleur2. De plus le protocole lui-même semble souffrir de problèmes de sécurité.Dans l'état actuel de cette norme, il semble plus prudent de ne confier à Z-Wave que des tâches domotiques limitées aux éléments dont le dysfonctionnement ou le piratage ne pose pas de problème.
\subsubsubsection{Capteurs utilisés}
*FIBARO Smoke Sensor:

Ce détecteur est très sensible à la fumée, mais pas juste àa la fumée. Certains matériaux brûlent avant de capter la fumée sous haute température . Voilà pourquoi les ingénieurs de  FIBARO ont décidé d'inclure une protection supplémentaire: un capteur de fumée sous la forme d'un capteur de température . Si la quantité de fumée n'est pas assez suffisante pour déclencher l'alarme , l'appareil sera toujours en mesure de détecter la menace; la découverte d'un changement rapide de température provoquée par le feu, le changement rapide de la température ou lorsque la température dépasse les 54°C. Ceci est alors suffisant pour le capteur de fumée pour découvrir la menace et signaler les utilisateurs à ce sujet. 


-    Détecteur de fumée et de température.


-    Nouvelle version à la norme Norme EN14604


-    Fonction "boite noire" (mémoire des événements).


-    Sans-fil Z-Wave+.


-    Design et matériaux nobles.


-    Compact avec seulement 6,5x2,8cm


-    Sensibilité réglable.


-    Bouton de configuration et d'indication multicolore.


-    Alimentation par pile fournie 


-    Autonomie de 3 ans env. (peut varier suivant les paramètres de réveil et de transmission des températures).


-    Garantie 2 ans

FIBARO Flood Sensor:

LA capteur d'innondation FIBARO , une taille compacte et une grande variété de fonctions supplémentaires . Ce capteur est tout simplement remarquable ! Ce dispositif unique peut vous garantir une sécurité optimale . Avec sa technologie de pointe et de précision , le capteur d'innondation Fibaro vous alerte de crue menaçante , ou un changement radical de température . Tout en étant sans entretien et sans la nécessité d'une installation professionnelle .


%Une photo de raspberry pi2 doit être mise ici
%Mamoune: Dans cette partie il est conseillé de prendre une photo représentant la conception 
\subsection*{le Logiciel (domoticz)}
Domoticz est un logiciel libre de gestion de domotique qui a pour but d’être exécutable sur un grand nombre de machines différente. Et ce qui fait son principal atout, c’est que le Raspberry Pi fait partie des machines sur lesquelles il peut tourner ! Ce qui permet donc d’en faire une machine dédiée aux opérations domotiques à prix très réduit. 
%		C)	l'application
% 
%III)	Comment le mettre en place Marc
%IV)	présentation appronfondie des composants.
%V)		Nos diffficultés difficulté
%VII)	Conclusion
%
%
%Ne pas oublier le glossaire.
%
% chaque partie est à ecrire dans un fichier indépandant du main
% il suffit de rajouter la ligne suivante pour inclure ces écrits
%\input ./maPartie.tex %sans saut de ligne
%
%%%%%%%%%%%%%%%%%%%


%%%%%%%%%%%%%%%%%%%%%%%%%%%%%%%%%%%%%%%%%%%%%%%%%%%%%%

\chapter{Introduction}

Ce projet se présente comme une très bonne expérience sur le plan théorique et pratique, car il permet de concevoir une première approche sur le monde des objets connectés et plus spécialement en domotiques. Il constitue aussi une occasion unique pour mettre en évidence le cumul des connaissances sue nous avons acquis tout au long de notre formation spécialisée en sécurité ubiquitaire.


De ce fait, notre binôme a décidé de réaliser un projet concernant la domotique. Toutefois, il est nécessaire de définir ce qu'est la domotique. La domotique est l'ensemble des techniques de l'électronique, de physique du bâtiment, d'automatisme, de l'informatique et des télécommunications utilisées dans les bâtiments, plus ou moins « interopérables » et permettant de centraliser le contrôle des différents systèmes et sous-systèmes de la maison et de l'entreprise (chauffage, volets roulants, porte de garage, 
portail d'entrée, prises électriques, etc.). La domotique vise à apporter des solutions techniques pour répondre aux besoins de 
confort (gestion d'énergie, optimisation de l'éclairage et du chauffage), de sécurité (alarme) et de communication (commandes à 
distance, signaux visuels ou sonores, etc.) que l'on peut retrouver dans les maisons, les hôtels, les lieux publics, etc.


Ce projet consiste alors à créer une petite centrale domotique grâce à une raspberry, permettant d'informer l'utilisateur sur certaines données captées grâce à des capteurs. Cette mini centrale permettera alors à l'utilisateur d'améliorer son confort et surtout sa sécurité. Ceci ne pourrait être que bénéfique envers les utilisateurs potentiels. La domotique est de plus en plus présente dans notre quotidien. Grâce à elle nous pouvons alors économiser de l'énergie ( gestion du chauffage, gestion de l'éclairage, gestion des volets ), augmenter l'autonomie des personnes handicapées ( assistance à l'ouverture des portes, des fenêtres, des volets, pilotage des appareils électriques, commande vocale ) ou encore améliorer la sécurité de nos habitations ( système d'alarme ). La mise en place d'un système domotisé peut se faire dès la construction d'un bâtiment ( norme KNX ), lors d'une rénovation, ou encore de façon ponctuelle ( norme X10 ).
%%%%%%%%%%%%%%%%%%%%%%%%partie III%%%%%%%%%%%%%%%%%%%%

\input ./III_MiseEnPlace.tex %sans saut de ligne

%%%%%%%%%%%%%%%%%%%%%%%%%%%%%%%%%%%%%%%%%%%%%%%%%%%%%%%


\end{document}
